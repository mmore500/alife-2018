\section{Introduction}

It is the aim of artificial life research to realize engineered systems that exhibit properties of biological life in order to study them, but also with an eye towards applications such as fault-tolerant infrastructure, artificial intelligence, etc.


Open-ended evolution

evolutionary transitions

fraternal transition of individuality

* definition

* example -- social insects: hierarchical

in order to we need to design a framework in which we expect such transitions to occur and, importantly, occur in a way that is straightforward to detect.

To this end, we introduce the dishtiny framework.
seeks to achieve this by making organisms explicitly register in cooperating groups (``channels'') to cooperate spatiotemporally to maximize the harvest of a resource.
Then, detecting transitions of individuality simply by looking for the presence of resource-sharing and reproductive division of labor among organisms sharing a channel and the absence among organisms *not* sharing a channel.
We believe that one of the most interesting properties of our system this abstraction is organized so that it can scale to an arbitrary number of hierarchical levels.
We see the ability to select for an arbitrary number of evolutionary transitions as an important step to achieving open-ended scale in artificial life that more closely resembles that of biological life.
