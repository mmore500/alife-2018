\begin{abstract}
% Abstract length should not exceed 250 words

% @CAO: This first sentence was way too convoluted.
%Evolutionary transitions in individuality, the emergence of new replicating entities from the unition of existing entities, are profound events in biological evolutionary history essential to the derivation of complex and diverse forms of life.
The emergence of new replicating entities from the union of existing entities represent some of the most profound events in natural evolutionary history. Facilitating such evolutionary transitions in individuality is essential to the derivation of the most complex forms of life.
As such, understanding these transitions is critical for building artificial systems capable of open-ended evolution.
Alas, these transitions are challenging to induce or detect, even with computational organisms.
Here, we introduce the DISHTINY (DIStributed Hierarchical Transitions of IndividualitY) platform, which
%focuses on this goal by providing 
provides simple cell-like organisms with the ability and incentive to unite into new individuals in a manner that can continue to scale to subsequent transitions.
% @CAO Next line is new:
The system is designed to encourage these transitions so that they can be studied:
organisms that coordinate spatiotemporally can maximize the rate of resource harvest, which is closely linked to their reproductive ability.
We demonstrate the hierarchical emergence of multiple levels of individuality among simple cell-like organisms that evolve parameters for manually-designed strategies.
During evolution, we observe reproductive division of labor and close cooperation between cells, including resource-sharing, aggregation of resource endowments for propagules, and emergence of an apoptosis response to somatic mutation.
% @CAO Another new line, if you think it's helpful:
Low-level organisms lose their ability to survive independently as transitions occur.
While a few replicate populations evolved selfish behaviors, many evolved to direct their resources toward low-level groups (behaving like multi-cellular individuals), and many others evolved to direct their resources toward high-level groups (acting as larger-scale multi-cellular individuals).
%We used ecological competition experiments
Finally, we demonstrated that genotypes that encode higher-level individuality consistently outcompete those that encode lower-level individuality. 

\end{abstract}
