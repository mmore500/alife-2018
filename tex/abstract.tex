\begin{abstract}
% Abstract length should not exceed 250 words

Evolutionary transitions in individuality, the emergence of new replicating entities from the unition of existing entities, are profound events in biological evolutionary history essential to the derivation of complex and diverse forms of life.
As such, understanding these transitions is critical for building artificial systems capable of open-ended evolution.
Alas, these transitions are challenging to induce and detect, even with computational organisms.
Here, we introduce the DISHTINY (DIStributed Hierarchical Transitions of IndividualitY) platform, which focuses on this goal by providing simple cell-like organisms the ability and incentive to unite into new individuals in a manner that can further scale to subsequent transitions.
Low-level cells must coordinate spatiotemporally to maximize the rate of resource harvest, which is closely linked to their reproductive ability.
We demonstrate the hierarchical emergence of multiple levels of individuality among simple cell-like organisms that evolve parameters for manually-designed strategies.
During evolution, we observe reproductive division of labor and close cooperation between cells, including resource-sharing, aggregation of resource endowments for propagules, and emergence of an apoptosis response to somatic mutation.
A few replicate populations evolved to behave selfishly, many evolved to direct their resources toward low-level groups (behaving like multi-cellular individuals), and many others evolved to direct their resources toward high-level groups (acting as much larger multi-cellular individuals).
We used ecological competition experiments to demonstrate that genotypes that encode higher-level individuality consistently outcompete those that encode lower-level individuality.

\end{abstract}
