\begin{abstract}
% Abstract length should not exceed 250 words

Evolutionary transitions in individuality
%have been key to the complexification and diversification of biological life.
represent some of the most profound changes leading to more complex and diverse forms of life.
As such, understanding these transitions %are thought to be a key factor with respect to the question
is critical for building artificial systems capable
of open-ended evolution.
Alas, these transitions tend to occur slowly and are challenging to detect, even with computational organisms.
%In order to study evolutionary transitions of individuality, we must devise a system in which we expect such transitions to occur in a detectable manner.
%To this end
Here, we introduce the DISHTINY (DIStributed Hierarchical Transitions of IndividualitY) platform, which focuses on this goal by providing simple organisms the ability and incentive to unite into new individuals in a manner that can further scale to subsequent transitions.
Low-level organisms must explicitly register in cooperating groups that coordinate spatiotemporally to maximize the harvest of a resource and link their reproductive ability.
%We lay out the design of DISHTINY and discuss its scalability.
%We use ecological competition and evolutionary experiments to demonstrate selection for and
We demonstrate the hierarchical
emergence of multiple levels of individuality using simple organisms that evolve parameters for manually-designed strategies.
In evolutionary experiments, organisms initially behaved selfishly, but soon directed their resources toward low-level groups (behaving like multi-cellular individuals), before ultimately forming high level groups and acting as much larger individuals.
During evolution, we observe reproductive division of labor and close cooperation between individuals registered to the same cooperating group, including resource-sharing and emergence of an apoptosis response to somatic mutation.
Indeed, we used ecological competition experiments to demonstrate that genotypes that encode higher-level individuality consistently  outcompete those that encode lower-level individuality.

\end{abstract}
