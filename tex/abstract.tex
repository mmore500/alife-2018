\begin{abstract}
% Abstract length should not exceed 250 words

Evolutionary transitions of individuality have been key to the complexification and diversification of biological life.
Realizing and studying such transitions are thought to be a key factor with respect to the question of open-ended evolution.
In order to study evolutionary transitions of individuality, we must devise a system in which we expect such transitions to occur in a detectable manner.
To this end, we introduce the DISHTINY (DIStributed Hierarchical Transitions of IndividualitY) platform, which seeks to achieve this goal by explicitly registering organisms in cooperating groups that coordinate spatiotemporally to maximize the harvest of a resource.
We lay out the design of DISHTINY and discuss its scalability.
Then, we use ecological competition and evolutionary experiments to demonstrate selection for and emergence of high-level individuality using simple organisms that evolve parameters for manually-designed strategies.
In evolutionary experiments, we observe reproductive division of labor and close cooperation between individuals registered to the same cooperating group, including resource-sharing and emergence of an apoptosis response to somatic mutation.
Ecological competition experiments demonstrate that genotypes that encode higher-level individuality outcompete those that encode lower-level individuality.

\end{abstract}
